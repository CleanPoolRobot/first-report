\chapter{Introdução}
O ser humano busca cada vez mais produtos que facilitem e tornem tarefas 
diárias mais eficientes. Essa busca gerou ao longo da história inovações e 
aprimoramento de produtos por meio de automações. Historicamente, a automação 
iniciou-se na revolução industrial do século XIX, porém na forma residencial, 
ela surgiu apenas na década de 70, quando foram lançados nos EUA os primeiros 
módulos inteligentes que utilizavam a rede elétrica como canal de comunicação 
entre os diversos dispositivos de automação \cite{sra}. A partir deste início 
houve um aumento do desenvolvimento de dispositivos dedicados (embarcados) por 
meio da utilização de microprocessadores e microcontroladores. De acordo com 
Miratori e Bó, o mercado de automação residencial dos Estados Unidos 
movimentou, até 2008, aproximadamente U\$\$ 10,5 bilhões. Além da praticidade 
e conforto, a segurança e eficiência também são fatores importantes para a 
escolha dos produtos residenciais. Dentre algumas das automações residenciais 
modernas destacam-se os aspiradores de pó, que são capazes de realizar a 
aspiração dos resíduos de forma rápida de ambientes sem a dispersão dos 
resíduos. Tendo em vista a necessidade de aspiração em piscinas e a crescente 
onda de automação residencial, observou-se a oportunidade de produzir um 
equipamento para realizar tal tarefa automática de piscinas \cite{kanno2014}.

\section{Contextualização}
O Brasil é o segundo colocado no ranking mundial de países com maior número de 
piscinas, com 1,8 milhão de unidades instaladas (sendo 80\% particulares), 
atrás apenas dos Estados Unidos. O país fatura anualmente cerca de R\$ 4,2 
bilhões com a construção e manutenção de piscinas (Francal Feira, 2013). Desta 
forma, atividades relacionadas ao uso da piscina estão cada vez mais 
requisitadas, entre elas a limpeza. A limpeza do fundo da piscina é um processo
que demanda tempo e esforço físico na sua realização . Uma pesquisa realizada 
em  março de 2014 no Estado de São Paulo apontou que o valor médio mensal da 
mão de obra para limpeza de piscinas de até 60 métros cúbicos, com equipamento exceto 
produtos para limpeza e manutenção,  foi de R\$ 303,00. Um valor que se torna 
cada vez mais significativo se imaginarmos piscinas maiores \cite{datafolha2014}. 
Com isso, várias empresas vêm investindo no desenvolvimento de robôs que possam
realizar aspiração de piscinas.No Brasil existem basicamente 3 tipos de 
produtos que realizam a tarefa de limpeza: aspiradores manuais que são os mais 
comuns no mercado, aspiradores robóticos, totalmente automatizados porém com 
alto custo, e aspiradores autônomos porém pouco eficazes \cite{miura2009}.

\section{Justificativa}
Considerando o alto custo de aquisição de equipamentos eficientes, a grande 
demanda por serviços de higienização de piscinas, e o fato de que no Brasil não
há empresas que desenvolvam este produto (há apenas revendedoras), propõe-se o 
desenvolvimento de um produto concorrente. Um robô  que possa realizar a tarefa
de aspiração de  modo satisfatório e com um valor de mercado mais acessível. O 
\textit{Clean Pool Robot} permitirá a aplicação do conceito de automação a 
ambientes residenciais e a redução dos gastos com a contratação de um serviço 
terceirizado. O robô ainda auxilia na privacidade, pois evita o acesso de 
pessoas estranhas à residências ou condomínios. 

\section{Objetivos}
Esse trabalho visa alcançar os objetivos, geral e específicos, apresentados a 
seguir:

\subsection{Objetivo Geral}
O projeto tem por finalidade o desenvolvimento de um robô que realiza a limpeza
do fundo de piscinas.

\subsection{Objetivos Específicos}
Para alcançar o objetivo proposto o projeto irá produzir um protótipo para:
\begin{itemize}
  \item Realizar aspiração automática dos resíduos decantados por meio da sucção e filtragem;
  \item Submergir de forma independente;
  \item Movimentar-se ao longo do fundo da piscina.
 
\end{itemize}

\section{Análise dos Concorrentes}
Um dos primeiros aparelhos utilizados na higienização de piscinas é o Limpa 
Piscina Automático. Este aparelho utiliza o sistema de filtragem da piscina 
como fonte de energia. Fácil de instalar, é necessário apenas ligá-lo à tomada
de aspiração, sem que seja necessário a instalação de quaisquer componentes. 
A potência da bomba da piscina permite que o aparelho se desloque 
automaticamente, de forma a aspirar os resíduos, que ficam retidos no filtro 
da bomba.


 
