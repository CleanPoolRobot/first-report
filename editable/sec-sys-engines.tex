\section{Os Motores}
Serão utilizados dois motores elétricos para rotacionar os rolos de limpeza. A transmissão do torque para os rolos será por engrenagens. O modelo de motor a ser usado é o: \textsf{Micro Motor DC 12V 18200RPM 390.40Gf.cm} que possui as seguintes especificações:

Especificações técnicas do produto:
\begin{itemize}
\item Corrente: 1350,00 mA;
\item Potência: 6,40 W;
\item Tensão Nominal: 12,00 V;
\item Tensão Operacional: 6V ~ 18V;
\item Torque: 390,40 Gf.cm;
\item Velocidade: 18200 RPM;
\item Peso: 213g.
\end{itemize}

Especificações técnicas de máximo rendimento:
\begin{itemize}
\item Rotação: 15700rpm;
\item Corrente: 6.8A;
\item Torque 390gf.cm;
\item Potência: 6.4w;
\item Rendimento: 77.5%;
\item Torque de Partida: 1.3kgf.cm.
\end{itemize}

\subsection{O Dimensionamento}
Para a escolha dos motores foi utilizado 2 critérios: rotação das escovas e a massa das mesmas. O motor deverá ser capaz de rotacionar as escovas entre 2000 e 3000rpm e considerando que elas tem massa igual a 450g ter torque suficiente para realizar o giro.
